\section*{ВВЕДЕНИЕ}
\addcontentsline{toc}{section}{ВВЕДЕНИЕ}

В эпоху цифровизации и автоматизации процессов, особое значение приобретает развитие технологий компьютерного зрения. Одним из перспективных направлений является создание систем, способных распознавать и классифицировать объекты на изображениях с высокой точностью и скоростью. Это становится возможным благодаря применению адаптивных нечётких нейронных сетей, которые могут анализировать цветовые характеристики объектов и адаптироваться к различным условиям освещения и цветовым профилям. Такие системы находят широкое применение в разнообразных областях, включая медицинскую диагностику, системы безопасности и многие другие.

Современные достижения в области искусственного интеллекта и машинного обучения открывают новые горизонты в разработке программного обеспечения, способного выполнять задачи, ранее считавшиеся недостижимыми для автоматизированных систем. Разработка стационарного приложения с адаптивной нечёткой нейронной сетью для распознавания объектов на изображениях по цветовым характеристикам является одним из таких прорывных проектов. Стационарные системы обладают рядом преимуществ, включая повышенную вычислительную мощность и расширенные возможности для интеграции с другими технологическими решениями, что делает их идеальными для сложных задач компьютерного зрения.

\emph{Цель настоящей работы} – разработка приложения на базе адаптивной нечёткой нейронной сети для распознавания и классификации объектов на изображениях по цветовым характеристикам. Приложение предназначено для улучшения процессов анализа изображений, обеспечивая высокую точность и скорость обработки данных. Особое внимание уделяется разработке алгоритмов, способных адаптироваться к изменяющимся условиям и разнообразию цветовых профилей объектов, что делает его применимым в различных областях, от медицинской диагностики до систем безопасности. Для достижения поставленной цели необходимо решить \emph{следующие задачи:}
\begin{itemize}
\item провести анализ предметной области;
\item разработать концептуальную модель приложения;
\item спроектировать приложение;
\item реализовать приложение.
\end{itemize}

\emph{Структура и объем работы.} Отчет состоит из введения, 4 разделов основной части, заключения, списка использованных источников, 2 приложений. Текст выпускной квалификационной работы равен \formbytotal{lastpage}{страниц}{е}{ам}{ам}.

\emph{Во введении} сформулирована цель работы, поставлены задачи разработки, описана структура работы, приведено краткое содержание каждого из разделов.

\emph{В первом разделе} на стадии описания технической характеристики предметной области приводится сбор информации о используемой архитектур.

\emph{Во втором разделе} на стадии технического задания приводятся требования к разрабатываемому приложению.

\emph{В третьем разделе} на стадии технического проектирования представлены проектные решения для приложения.

\emph{В четвертом разделе} приводится список классов и их методов, использованных при разработке сайта, производится тестирование разработанного сайта.

В заключении излагаются основные результаты работы, полученные в ходе разработки.

В приложении А представлен графический материал.
В приложении Б представлены фрагменты исходного кода. 
