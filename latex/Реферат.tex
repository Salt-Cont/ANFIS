\abstract{РЕФЕРАТ}

Объем работы равен \formbytotal{lastpage}{страниц}{е}{ам}{ам}. Работа содержит \formbytotal{figurecnt}{иллюстраци}{ю}{и}{й}, \formbytotal{tablecnt}{таблиц}{у}{ы}{}, \arabic{bibcount} библиографических источников и \formbytotal{числоПлакатов}{лист}{}{а}{ов} графического материала. Количество приложений – 2. Графический материал представлен в приложении А. Фрагменты исходного кода представлены в приложении Б.

Перечень ключевых слов: компьютерное зрение, язык Python, средства MATLAB, приложение, нечёткая нейронная сеть, система, информатизация, автоматизация, информационные технологии, классы, ANFIS, база данных, компонент, нечёткая логика, модуль, сущность, метод, пользователь, нейросеть, стационарное приложение.

Объектом разработки является стационарное приложение,  обеспечивающее возможность распознавания объектов по цветовым характеристикам на основе нечётких нейронных сетей.

Целью выпускной квалификационной работы является проведение глубокого анализа и исследования передовых технологий в области компьютерного зрения и искусственного интеллекта. Работа фокусируется на применении нечётких нейронных сетей, что позволяет значительно повысить точность и эффективность процесса распознавания объектов.

В ходе разработки приложения был выполнен ряд ключевых шагов: определение и структурирование основных сущностей через создание информационных модулей, применение классов и методов для управления данными в соответствии с предметной областью и для обеспечения надёжности функционирования приложения. Была спроектирована архитектура нечёткой нейронной сети и базы данных, а также разработан пользовательский интерфейс, облегчающий взаимодействие с приложением.

Для создания веб-сайта были использованы возможности языка программирования Python в сочетании с инструментарием библиотек MATLAB, что позволило реализовать необходимые функциональные возможности.

\selectlanguage{english}
\abstract{ABSTRACT}

The amount of work is equal to \formbytotal{lastpage} pages. The work contains \formbytotal{figurecnt} illustrations, \formbytotal{tablecnt} tables, \arabic{bibcount} bibliographic sources and \formbytotal{числоПлакатов} sheets of graphic material. Number of applications – 2. Graphic material is presented in Appendix A. Source code fragments are presented in Appendix B.

List of keywords: computer vision, Python language, MATLAB tools, application, fuzzy neural network, system, informatization, automation, information technology, classes, ANFIS, database, component, fuzzy logic, module, entity, method, user, neural network, stationary application.

The object of development is a stationary application that provides the ability to recognize objects by color characteristics based on fuzzy neural networks.

The purpose of the final qualifying work is to conduct in-depth analysis and research of advanced technologies in the field of computer vision and artificial intelligence. The work focuses on the use of fuzzy neural networks, which can significantly improve the accuracy and efficiency of the object recognition process.

During the development of the application, a number of key steps were completed: defining and structuring the main entities through the creation of information modules, using classes and methods to manage data in accordance with the subject area and to ensure the reliability of the application. The architecture of the fuzzy neural network and database was designed, and a user interface was developed to facilitate interaction with the application.

To create the website, the capabilities of the Python programming language were used in combination with the tools of the MATLAB libraries, which made it possible to implement the necessary functionality.
\selectlanguage{russian}
